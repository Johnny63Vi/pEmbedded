\chapter{Inleiding}


In dit practicum maak je kennis met het programmeren van een Embedded System, in dit geval een 
BBC Micro:bit. Het doel is om je een idee te geven waar je bij de differentiatie ‘Network \& Systems Engineering’ mee te maken krijgt. Daar leer je software te schrijven om hardware aan te sturen.

In dit practicum houden we het simpel. We proberen je analytische vaardigheden te testen, te prikkelen en te ontwikkelen. Je bestudeert de gegeven voorbeelden om de eigenschappen van de hardware en software te ontdekken. Dit doe je tijdens het practicum, op deze manier kun je de docent vragen stellen als het niet duidelijk is en zo kan de docent zien of je met de stof bezig bent.  

Beantwoord vragen Specifiek, Meetbaar, Acceptabel,  Realistisch en op Tijd (SMART). 
Kijk niet alleen wat iets doet, maar vooral hoe iets werkt en waarom. HBO is (ook) analyseren!
De voorbeelden zijn geschreven in de taal C (of C++, afgeleid van C). De taal C is de meest gebruikte taal voor embedded systems.

Je leert nu nog geen C-programmeren, dat gebeurt bij NSE in het 2e semester. In het 2e jaar gaan we bij NSE uitgebreid verder met C++ en Object Oriented Programming.

We vragen je nu alleen om de code te lezen en te analyseren. Als je Java programmeren in Periode 1 met succes hebt afgerond, dan moet dat lukken. 

De programma’s die je moet maken kun je in elkaar zetten door de code uit de voorbeelden op een slimme manier bij elkaar te kopiëren en te plakken. Zo doe je dat op beginners niveau.

Het belangrijkste doel van het practicum is om een ‘smaakje’ te krijgen van NSE. 
Het practicum ondersteunt op een aantal punten de theorie (zie de leerdoelen op Blackboard). 

Er is in principe géén klassikale instructie. Je werkt zelfstandig aan de hand van deze practicum handleiding. Je kunt wel vragen stellen aan een docent of assistent. Je kunt ook vragen stellen over hoe je de Microbit in ‘The Challenge’ gebruikt, maar je moet zelf de software schrijven.

Er zit geen toetsing in het practicum. De toetsing zit deels in de theorietoets en betreft de opdrachten uit de handleiding. Een aantal opdrachten kunnen via blackboard ingeleverd worden. Aan de hand van de ingeleverde opdrachten komt een aantekening te staan bij de 'The Challenge' en wordt meegewogen bij het advies dat gegeven wordt indien de NSE richting gekozen wordt. Verdere toetsing zit in jullie uitwerking van het Embedded deel van ‘The Challenge’.

In deze handleiding wordt in hoofdstuk \ref{chap:omgeving} de practicum omgeving besproken, In hoofdstuk \ref{chap:intr} worden de basis principes van het embedded programmeren besproken. In hoofdstuk \ref{chap:bewSen} wordt de bewegingssensor van de microbit besproken. In de hoofdstukken \ref{chap:matrix} en \ref{chap:biw} wordt de matrix en bitwise bewerkingen gedaan.
In de appendix, worden een van een aantal problemen een oplossing gegeven, verder wordt ingegaan op mogelijke toepassingen bij de challenge.

Voor het practicum heb je een \textbf{eigen laptop, een BBC Microbit en Arduino software} nodig 
(zie de materiaallijst; de Microbit Go kit kost ongeveer \euro{}20). 